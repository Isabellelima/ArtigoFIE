%% bare_conf.tex
%% V1.4b
%% 2015/08/26
%% by Michael Shell
%% See:
%% http://www.michaelshell.org/
%% for current contact information.
%%
%% This is a skeleton file demonstrating the use of IEEEtran.cls
%% (requires IEEEtran.cls version 1.8b or later) with an IEEE
%% conference paper.
%%
%% Support sites:
%% http://www.michaelshell.org/tex/ieeetran/
%% http://www.ctan.org/pkg/ieeetran
%% and
%% http://www.ieee.org/

%%*************************************************************************
%% Legal Notice:
%% This code is offered as-is without any warranty either expressed or
%% implied; without even the implied warranty of MERCHANTABILITY or
%% FITNESS FOR A PARTICULAR PURPOSE! 
%% User assumes all risk.
%% In no event shall the IEEE or any contributor to this code be liable for
%% any damages or losses, including, but not limited to, incidental,
%% consequential, or any other damages, resulting from the use or misuse
%% of any information contained here.
%%
%% All comments are the opinions of their respective authors and are not
%% necessarily endorsed by the IEEE.
%%
%% This work is distributed under the LaTeX Project Public License (LPPL)
%% ( http://www.latex-project.org/ ) version 1.3, and may be freely used,
%% distributed and modified. A copy of the LPPL, version 1.3, is included
%% in the base LaTeX documentation of all distributions of LaTeX released
%% 2003/12/01 or later.
%% Retain all contribution notices and credits.
%% ** Modified files should be clearly indicated as such, including  **
%% ** renaming them and changing author support contact information. **
%%*************************************************************************


% *** Authors should verify (and, if needed, correct) their LaTeX system  ***
% *** with the testflow diagnostic prior to trusting their LaTeX platform ***
% *** with production work. The IEEE's font choices and paper sizes can   ***
% *** trigger bugs that do not appear when using other class files.       ***                          ***
% The testflow support page is at:
% http://www.michaelshell.org/tex/testflow/



\documentclass[conference]{IEEEtran}
% Some Computer Society conferences also require the compsoc mode option,
% but others use the standard conference format.
%
% If IEEEtran.cls has not been installed into the LaTeX system files,
% manually specify the path to it like:
% \documentclass[conference]{../sty/IEEEtran}





% Some very useful LaTeX packages include:
% (uncomment the ones you want to load)


% *** MISC UTILITY PACKAGES ***
%
%\usepackage{ifpdf}
% Heiko Oberdiek's ifpdf.sty is very useful if you need conditional
% compilation based on whether the output is pdf or dvi.
% usage:
% \ifpdf
%   % pdf code
% \else
%   % dvi code
% \fi
% The latest version of ifpdf.sty can be obtained from:
% http://www.ctan.org/pkg/ifpdf
% Also, note that IEEEtran.cls V1.7 and later provides a builtin
% \ifCLASSINFOpdf conditional that works the same way.
% When switching from latex to pdflatex and vice-versa, the compiler may
% have to be run twice to clear warning/error messages.






% *** CITATION PACKAGES ***
%
%\usepackage{cite}
% cite.sty was written by Donald Arseneau
% V1.6 and later of IEEEtran pre-defines the format of the cite.sty package
% \cite{} output to follow that of the IEEE. Loading the cite package will
% result in citation numbers being automatically sorted and properly
% "compressed/ranged". e.g., [1], [9], [2], [7], [5], [6] without using
% cite.sty will become [1], [2], [5]--[7], [9] using cite.sty. cite.sty's
% \cite will automatically add leading space, if needed. Use cite.sty's
% noadjust option (cite.sty V3.8 and later) if you want to turn this off
% such as if a citation ever needs to be enclosed in parenthesis.
% cite.sty is already installed on most LaTeX systems. Be sure and use
% version 5.0 (2009-03-20) and later if using hyperref.sty.
% The latest version can be obtained at:
% http://www.ctan.org/pkg/cite
% The documentation is contained in the cite.sty file itself.






% *** GRAPHICS RELATED PACKAGES ***
%
\ifCLASSINFOpdf
  % \usepackage[pdftex]{graphicx}
  % declare the path(s) where your graphic files are
  % \graphicspath{{../pdf/}{../jpeg/}}
  % and their extensions so you won't have to specify these with
  % every instance of \includegraphics
  % \DeclareGraphicsExtensions{.pdf,.jpeg,.png}
\else
  % or other class option (dvipsone, dvipdf, if not using dvips). graphicx
  % will default to the driver specified in the system graphics.cfg if no
  % driver is specified.
  % \usepackage[dvips]{graphicx}
  % declare the path(s) where your graphic files are
  % \graphicspath{{../eps/}}
  % and their extensions so you won't have to specify these with
  % every instance of \includegraphics
  % \DeclareGraphicsExtensions{.eps}
\fi
% graphicx was written by David Carlisle and Sebastian Rahtz. It is
% required if you want graphics, photos, etc. graphicx.sty is already
% installed on most LaTeX systems. The latest version and documentation
% can be obtained at: 
% http://www.ctan.org/pkg/graphicx
% Another good source of documentation is "Using Imported Graphics in
% LaTeX2e" by Keith Reckdahl which can be found at:
% http://www.ctan.org/pkg/epslatex
%
% latex, and pdflatex in dvi mode, support graphics in encapsulated
% postscript (.eps) format. pdflatex in pdf mode supports graphics
% in .pdf, .jpeg, .png and .mps (metapost) formats. Users should ensure
% that all non-photo figures use a vector format (.eps, .pdf, .mps) and
% not a bitmapped formats (.jpeg, .png). The IEEE frowns on bitmapped formats
% which can result in "jaggedy"/blurry rendering of lines and letters as
% well as large increases in file sizes.
%
% You can find documentation about the pdfTeX application at:
% http://www.tug.org/applications/pdftex





% *** MATH PACKAGES ***
%
%\usepackage{amsmath}
% A popular package from the American Mathematical Society that provides
% many useful and powerful commands for dealing with mathematics.
%
% Note that the amsmath package sets \interdisplaylinepenalty to 10000
% thus preventing page breaks from occurring within multiline equations. Use:
%\interdisplaylinepenalty=2500
% after loading amsmath to restore such page breaks as IEEEtran.cls normally
% does. amsmath.sty is already installed on most LaTeX systems. The latest
% version and documentation can be obtained at:
% http://www.ctan.org/pkg/amsmath





% *** SPECIALIZED LIST PACKAGES ***
%
%\usepackage{algorithmic}
% algorithmic.sty was written by Peter Williams and Rogerio Brito.
% This package provides an algorithmic environment fo describing algorithms.
% You can use the algorithmic environment in-text or within a figure
% environment to provide for a floating algorithm. Do NOT use the algorithm
% floating environment provided by algorithm.sty (by the same authors) or
% algorithm2e.sty (by Christophe Fiorio) as the IEEE does not use dedicated
% algorithm float types and packages that provide these will not provide
% correct IEEE style captions. The latest version and documentation of
% algorithmic.sty can be obtained at:
% http://www.ctan.org/pkg/algorithms
% Also of interest may be the (relatively newer and more customizable)
% algorithmicx.sty package by Szasz Janos:
% http://www.ctan.org/pkg/algorithmicx




% *** ALIGNMENT PACKAGES ***
%
%\usepackage{array}
% Frank Mittelbach's and David Carlisle's array.sty patches and improves
% the standard LaTeX2e array and tabular environments to provide better
% appearance and additional user controls. As the default LaTeX2e table
% generation code is lacking to the point of almost being broken with
% respect to the quality of the end results, all users are strongly
% advised to use an enhanced (at the very least that provided by array.sty)
% set of table tools. array.sty is already installed on most systems. The
% latest version and documentation can be obtained at:
% http://www.ctan.org/pkg/array


% IEEEtran contains the IEEEeqnarray family of commands that can be used to
% generate multiline equations as well as matrices, tables, etc., of high
% quality.




% *** SUBFIGURE PACKAGES ***
%\ifCLASSOPTIONcompsoc
%  \usepackage[caption=false,font=normalsize,labelfont=sf,textfont=sf]{subfig}
%\else
%  \usepackage[caption=false,font=footnotesize]{subfig}
%\fi
% subfig.sty, written by Steven Douglas Cochran, is the modern replacement
% for subfigure.sty, the latter of which is no longer maintained and is
% incompatible with some LaTeX packages including fixltx2e. However,
% subfig.sty requires and automatically loads Axel Sommerfeldt's caption.sty
% which will override IEEEtran.cls' handling of captions and this will result
% in non-IEEE style figure/table captions. To prevent this problem, be sure
% and invoke subfig.sty's "caption=false" package option (available since
% subfig.sty version 1.3, 2005/06/28) as this is will preserve IEEEtran.cls
% handling of captions.
% Note that the Computer Society format requires a larger sans serif font
% than the serif footnote size font used in traditional IEEE formatting
% and thus the need to invoke different subfig.sty package options depending
% on whether compsoc mode has been enabled.
%
% The latest version and documentation of subfig.sty can be obtained at:
% http://www.ctan.org/pkg/subfig




% *** FLOAT PACKAGES ***
%
%\usepackage{fixltx2e}
% fixltx2e, the successor to the earlier fix2col.sty, was written by
% Frank Mittelbach and David Carlisle. This package corrects a few problems
% in the LaTeX2e kernel, the most notable of which is that in current
% LaTeX2e releases, the ordering of single and double column floats is not
% guaranteed to be preserved. Thus, an unpatched LaTeX2e can allow a
% single column figure to be placed prior to an earlier double column
% figure.
% Be aware that LaTeX2e kernels dated 2015 and later have fixltx2e.sty's
% corrections already built into the system in which case a warning will
% be issued if an attempt is made to load fixltx2e.sty as it is no longer
% needed.
% The latest version and documentation can be found at:
% http://www.ctan.org/pkg/fixltx2e


%\usepackage{stfloats}
% stfloats.sty was written by Sigitas Tolusis. This package gives LaTeX2e
% the ability to do double column floats at the bottom of the page as well
% as the top. (e.g., "\begin{figure*}[!b]" is not normally possible in
% LaTeX2e). It also provides a command:
%\fnbelowfloat
% to enable the placement of footnotes below bottom floats (the standard
% LaTeX2e kernel puts them above bottom floats). This is an invasive package
% which rewrites many portions of the LaTeX2e float routines. It may not work
% with other packages that modify the LaTeX2e float routines. The latest
% version and documentation can be obtained at:
% http://www.ctan.org/pkg/stfloats
% Do not use the stfloats baselinefloat ability as the IEEE does not allow
% \baselineskip to stretch. Authors submitting work to the IEEE should note
% that the IEEE rarely uses double column equations and that authors should try
% to avoid such use. Do not be tempted to use the cuted.sty or midfloat.sty
% packages (also by Sigitas Tolusis) as the IEEE does not format its papers in
% such ways.
% Do not attempt to use stfloats with fixltx2e as they are incompatible.
% Instead, use Morten Hogholm'a dblfloatfix which combines the features
% of both fixltx2e and stfloats:
%
% \usepackage{dblfloatfix}
% The latest version can be found at:
% http://www.ctan.org/pkg/dblfloatfix




% *** PDF, URL AND HYPERLINK PACKAGES ***
%
%\usepackage{url}
% url.sty was written by Donald Arseneau. It provides better support for
% handling and breaking URLs. url.sty is already installed on most LaTeX
% systems. The latest version and documentation can be obtained at:
% http://www.ctan.org/pkg/url
% Basically, \url{my_url_here}.




% *** Do not adjust lengths that control margins, column widths, etc. ***
% *** Do not use packages that alter fonts (such as pslatex).         ***
% There should be no need to do such things with IEEEtran.cls V1.6 and later.
% (Unless specifically asked to do so by the journal or conference you plan
% to submit to, of course. )


% correct bad hyphenation here
\hyphenation{op-tical net-works semi-conduc-tor}


\begin{document}
%
% paper title
% Titles are generally capitalized except for words such as a, an, and, as,
% at, but, by, for, in, nor, of, on, or, the, to and up, which are usually
% not capitalized unless they are the first or last word of the title.
% Linebreaks \\ can be used within to get better formatting as desired.
% Do not put math or special symbols in the title.
\title{Bare Demo of IEEEtran.cls\\ for IEEE Conferences}


% author names and affiliations
% use a multiple column layout for up to three different
% affiliations
\author{\IEEEauthorblockN{Michael Shell}
\IEEEauthorblockA{School of Electrical and\\Computer Engineering\\
Georgia Institute of Technology\\
Atlanta, Georgia 30332--0250\\
Email: http://www.michaelshell.org/contact.html}
\and
\IEEEauthorblockN{Homer Simpson}
\IEEEauthorblockA{Twentieth Century Fox\\
Springfield, USA\\
Email: homer@thesimpsons.com}
\and
\IEEEauthorblockN{James Kirk\\ and Montgomery Scott}
\IEEEauthorblockA{Starfleet Academy\\
San Francisco, California 96678--2391\\
Telephone: (800) 555--1212\\
Fax: (888) 555--1212}}

% conference papers do not typically use \thanks and this command
% is locked out in conference mode. If really needed, such as for
% the acknowledgment of grants, issue a \IEEEoverridecommandlockouts
% after \documentclass

% for over three affiliations, or if they all won't fit within the width
% of the page, use this alternative format:
% 
%\author{\IEEEauthorblockN{Michael Shell\IEEEauthorrefmark{1},
%Homer Simpson\IEEEauthorrefmark{2},
%James Kirk\IEEEauthorrefmark{3}, 
%Montgomery Scott\IEEEauthorrefmark{3} and
%Eldon Tyrell\IEEEauthorrefmark{4}}
%\IEEEauthorblockA{\IEEEauthorrefmark{1}School of Electrical and Computer Engineering\\
%Georgia Institute of Technology,
%Atlanta, Georgia 30332--0250\\ Email: see http://www.michaelshell.org/contact.html}
%\IEEEauthorblockA{\IEEEauthorrefmark{2}Twentieth Century Fox, Springfield, USA\\
%Email: homer@thesimpsons.com}
%\IEEEauthorblockA{\IEEEauthorrefmark{3}Starfleet Academy, San Francisco, California 96678-2391\\
%Telephone: (800) 555--1212, Fax: (888) 555--1212}
%\IEEEauthorblockA{\IEEEauthorrefmark{4}Tyrell Inc., 123 Replicant Street, Los Angeles, California 90210--4321}}




% use for special paper notices
%\IEEEspecialpapernotice{(Invited Paper)}




% make the title area
\maketitle

% As a general rule, do not put math, special symbols or citations
% in the abstract
\begin{abstract}

Educational Robotics (ER) has revealed several benefits in the educational context, not only helping the teaching of disciplines, but also making possible the development of several abilities, such as teamwork, problem-solving, and creativity. Among various robotics kits, LEGO Robotics has been shown one of the best results considering some evaluated criteria (modularity level, hardware, curriculum, price, etc.). Some studies analyze the teaching practices, some compare technologies, and others evaluate the kits in a pedagogical way. However, it is essential to investigate all these contexts together in order to improve the impact produced by the ER in education and to know the best teaching practices associated with the most powerful technologies. The objective of this work is to identify: a) environments and programming languages adopted in the LEGO Robotics context, b) educational practices applied during classes based on LEGO Robotics, and c) the educational levels in which robotics has been applied with positive results. To achieve these goals, we planned and carried out a systematic review of the literature. Our main findings are: a) the most widely used environment and programming language are LabVIEW along with the LEGO's block-based programming language, b) we identified LEGO Robotics is used for teaching programming, interdisciplinary contents, participation in tournaments, robotics, and computational thinking, c) LEGO Robotics is used with success by students of different levels, such as K-12, undergraduate, and graduated. Finally, we discuss some problems and limitations related to ER and point out that there is no standardization of teaching practices or methodologies for evaluating results, indicating that more research is needed to find the best scenario regarding technologies, methods, and target audience.
\end{abstract}

% no keywords




% For peer review papers, you can put extra information on the cover
% page as needed:
% \ifCLASSOPTIONpeerreview
% \begin{center} \bfseries EDICS Category: 3-BBND \end{center}
% \fi
%
% For peerreview papers, this IEEEtran command inserts a page break and
% creates the second title. It will be ignored for other modes.
\IEEEpeerreviewmaketitle



\section{Introduction}

Educational Robotics (ER) can be an educational tool indicated for teaching Science, Technology, Engineering and Mathematics (STEM) and Computer Science in all levels of education \cite{Rogers2}. Seymour Papert has pioneered the use of computer and robot as a way of learning. He believed on learning by making, where students were encouraged to discover and build knowledge through practice activities \cite{PAPERT}. Besides, several studies have demonstrated that ER is an approach to stimulate the development of varied skills, such as teamwork, logical reasoning, critical thinking and creativity \cite{Miller} \cite{Blanchard}. ER is multidisciplinary approach involving design, assembly and use of robots applying principles of engineering, computing, mathematics, physics, among other sciences.

The educational benefits of ER are provided by robotics kits and software in order to facilitate the programming of constructed robots. Currently, there are several robotics kits available for use, not only open source, such as Arduino but also commercials, such as LEGO Mindstorms and Fischertechnik. LEGO Mindstorms is widely accepted in schools around the world. According to Eigner et al., "LEGO Mindstorms is probably the most popular modular robotics kit, building on the famous bricks and with a wide range of applications" \cite{Eigner}. 

LEGO Mindstorms NXT or EV3 promotes education both with its physical part, while built blocks, and in the LabView visual programming environment, while programming the robots. According to Chambers et al. \cite{Chambers}, activities based on building blocks have helped students to understand the concepts of gears and a vehicle’s speed. The LabView visual programming environment provides programming learning in a fun and straightforward way, jointly stimulates the development of Computational Thinking \cite{Bers}.

Several research have analyzed the technological and pedagogical potential that robotics can offer for education. Kunduracıoğlu investigated how to use LabView with LEGO Mindstorms EV3 \cite{Kunduracıoglu}. Hirst et al. examined the programming environments for LEGO Mindstorms available until 2003 \cite{Hirst}. The research aimed to assist in choosing the environment for teaching LEGO Robotics. Eigner et al. investigated the development of communities that use robotics kits for educational purpose \cite{Eigner}. They pointed out that LEGO Robotics is one of the most popular and has the best support for education.

Although previous research have analyzed the pedagogical potential of robotics, there is still very little understanding of the impacts and contributions that ER offer, considering hardware, software, and educational practices in the same context. The existing studies focus on specific criteria and commonly ignore the applied teaching methodology during activities. This fact may interfere with the analysis as a whole. As far as we know, there is no systematic review of literature that analyzed the use of LEGO Robotics technology as an instrument for teaching, encompassing the analysis of the robotics kits combine its programming environments with teaching methodologies and practices at the same time.

In this context, it is necessary to know the best teaching practices associated with the most powerful technologies. The objective of this work is to identify: a) environments and programming languages adopted in the LEGO Robotics context; b) educational practices applied during classes based on LEGO Robotics; c) the educational levels in which robotics has been applied with positive results. To achieve these goals, we planned and carried out a systematic review of the literature (SRL) following the guidelines proposed by Kitchenham et al. \cite{Kitchenham} and Petersen et al. \cite{Petersen}. 

This paper is organized as follows. Section 2 presents the related works. Section 3 describes the process of SRL. Section 4 shows the results answering the research questions. Section 5 discusses the primary information observed during the SRL. Finally, Section 6 presents the conclusion.

%\hfill 

\section{Related Work}
R can promote students’ interest in education and scientific careers \cite{Bers}, \cite{Ospennikova}, \cite{Rogers} \cite{Lupetti}. These previous studies present discussions about teaching practices such as interdisciplinary teaching, teaching programming, and development of Computational Thinking.

Several studies argue over ER has remarkable education potential, but they only worry about the technological artifacts which support educational actions. Eigner et al. evaluated various robotics kit considering some criteria, such as quality of educational materials, development and extension possibilities, target audience, and modularity level \cite{Eigner}. All mentioned criteria are focused on educational resources in several robotics kits. Hirts et al. compared the programming environments for LEGO Mindstorms available until 2013 in order to assist in choosing the programming environment for teaching \cite{Hirst}. Kunduracıoglu aimed at providing detailed information on how to use LabView with LEGO Mindstorms EV3 \cite{Kunduracıoglu}.

To the best of our knowledge, there are no literature reviews focused on surveying the development environments for teaching with LEGO Robotics that contemplate pedagogical issues involved in the teaching process. Bers et al. \cite{Bers}, Ospennikova et al. \cite{Ospennikova}, Rogers et al. \cite{Rogers}, and Lupetti et al. \cite{Lupetti} are focused on applied teaching practices, while Eiger et al. \cite{Eigner} analyzed amount robotics kit. Besides, Eiger et al. used the popularity and availability of the manufacturers’ official materials as criteria to select the robotics kit for evaluation, without following a systematic cataloging process. In turn, Hist et al. compared the development environments for LEGO, but without obeying systematic process in 2003 \cite{Hirst}.

In the SRL, Cheng et al. present an investigation of the characteristics of ER in six different target groups (Preschool, Elementary, High School, Higher Education, Adults and Elderly) \cite{Cheng}. However, they did not evaluate considering the technological artifacts and pedagogical practices. Both aspects are essential to measure the effectiveness of learning through ER, especially if they are taken together in the same context. However, as far as we know, there is no RSL focused on technological artifacts and pedagogical practices in LEGO Robotics.


\section{Methodology}

This Section presents the methodology used in our SRL. We planned and executed a SRL following the guidelines proposed by Kitchenham et al. \cite{Kitchenham} and Petersen et al. \cite{Petersen}. This SRL methodology has been gaining attention in the Software Engineering community, because it follows a rigorous and auditable protocol allowing to  evaluate available studies relevant to a particular topic area and answer research questions. In addition, it enables the identification of gaps in current research, thereby opening possibilities for future work \cite{Kitchenham}.

The protocol was developed as proposed by \cite{Kitchenham}. The guideline was followed intent to guide the execution of the research plus reduce the possibility researcher bias. The detailed procedures are in the following subsections.

\subsection{Research Questions}

The research questions proposed in this SRL aim to gather information about programming environments and educational practices in  LEGO Robotics classes. The research questions (RQ) are: 

\textbf{RQ1:} What environments and programming languages have been used for teaching through LEGO Robotics? 
The objective of RQ1 is to identify environments and programming languages adopted in the LEGO Robotics context. 

\textbf{RQ2:} What has been taught using LEGO Robotics? 
 	The objective of RQ2 is to classify educational practices applied during classes based on LEGO Robotics. 

\textbf{RQ3:} What has been the target audience for the use of LEGO Robotics?
The objective of RQ3 is identify the educational levels in which robotics has been applied with positive results. 

\subsection{Conduct Search}

Planning is an imperative activity for a consistent SRL. After the definition of RQ,  the search string was constructed in the direction of supporting the search of the studies in the digital libraries. For this, we followed the guideline defined in the protocol:

\begin{itemize}

\item Identifying the main keywords of the research questions;
\item Identifying related words and synonyms for keywords;
\item Performing tests in the databases and checking the quality of the results. If necessary, remake the related words and synonyms for the keywords;
\item In the search string, use the keywords, related words, and synonyms that returned the number of studies aligned with the search questions; 
\item Using OR Boolean to connect synonyms/related words and AND Boolean to connect keywords;
\item Performing tests to evaluate the quality of the results. If necessary, remake the search string to obtain better results.

\end{itemize}

The main keywords were identified according to the research questions and the objective of the present study. The main keywords were Environments, LEGO and Robotics. The related words and synonyms for the keywords are shown in Table \ref{synonyms_keywords}.

\begin{table}[h]
\renewcommand{\arraystretch}{1.3}
\caption{Synonyms for the keywords}
\label{synonyms_keywords}
\centering
\begin{tabular}{|c||p{5cm}|}
\hline \textbf{Key words} &\textbf{Synonyms or related words}\\
\hline Environment & software OR programming OR environment OR package\\
\hline LEGO & mindstorms\\
\hline Robotic & -\\
\hline
\end{tabular}
\end{table}

After following the above-mentioned guideline, the search string was defined as:
 
\begin{center} 
\textbf{( (software OR programming OR environment OR package) AND (lego OR mindstorms)  AND robotic )}
\end{center}

After defining the search string, the next step was carried out the search in the ACM, IEEEXplore, ScienceDirect (Elsevier), and Scopus digital libraries. These libraries were chosen because they presented better results during the pilot test, besides they have been centralized the most publications related to Computer Science and Education. Table \ref{summary_library} shows the quantity of the studies returned in each digital library.


\begin{table}[h]
\renewcommand{\arraystretch}{1.3}
\caption{Summary of the studies returned in each digital library}
\label{summary_library}
\centering
\begin{tabular}{|c||c|}
\hline \textbf{Digital Library} &\textbf{Search Results}\\
\hline ACM	 & 36\\
\hline IEEE & 79\\
\hline ScienceDirect & 224\\
\hline Scopus & 1024\\
\hline \textbf{Total} & \textbf{1363}\\
\hline
\end{tabular}
\end{table}

\subsection{Screening of papers for inclusion and exclusion}

The selection of the studies considered inclusion and exclusion criteria. The inclusion criteria were:
\begin{itemize}
\item Studies whose address the use of LEGO Robotics in teaching;
\item Studies accessible electronically.
\end{itemize}

The exclusion criteria were:
\begin{itemize}
\item Studies whose are outside the field of research;
\item Studies that explore LEGO Robotics for teaching any subject, but it does not present evaluation, assessment, or validation;
\item Studies that use LEGO Robotics for teaching any subject, but it does not contemplate robot programming;
\item Non-peer reviewed studies, such as abstracts only, tutorials, editorials, slides, lectures, posters, panels, keynotes, and technical reports;
\item Peer-reviewed studies, but it was not published in journals, conferences, or workshops, such as doctoral thesis, books, and patents;
\item Studies whose language was not English; 
\item Studies published before 2013;
\item Studies electronically not accessible.
\item Duplicated and secondary studies.
\end{itemize}

Initially, we excluded duplicated and secondary studies. Then, the selection process was performed in two steps. In the first step, we read the titles and abstracts. The studies related to the mechatronics’ industry, using or not LEGO Robotics, were excluded. So, we selected 315 studies that may help to answer the research questions. In the second step, the introduction, methodology and conclusion sections of the selected studies in the first step were read. Studies which involved the development of LEGO Robotics laboratories were withdrawn. In this phase, 36 studies fit the context of the research with information capable of answering the research questions. These studies were read and carefully analyzed in order to extract as much information as necessary.

\section{Results}

This section details the results in our RSL. The research was conducted from October 2017 to April 2018. Initially, we will present the general results, including the distribution of publications by years, countries were the research has been done, types of publication vehicles and types of scientific research applied in the studies. Finally, the remaining sections answer each research question.

\subsection{General Results}
\begin{enumerate}
\item Publication year:FIGURE X presents the absolute number of studies by year of publication. The selected studies focus mainly on the years 2016 and 2015, both with 25\% publication each one. The years  2017, 2014 and 2013 presented a distribution of 16.7\%, 22\%, and 11.11\%, respectively. These numbers have been shown the interest of the scientific community with LEGO Robotics as an instrument for teaching since 2014. 
\item  Countries distribution: The result shows the United States of America was the country with the most significant number of research involving LEGO Robotics with 11 studies, following by Germany with four studies. South Africa, Brazil, and Spain published three studies each one, while the United Kingdom, Italy, and Cyprus published two studies each one. Canada, Chile, Slovenia, Georgia, Greece, India, Japan, Puerto Rico, Portugal, and Saudi Arabia have only one published study each one. The fact that the United States of America leads the research may be related to the country's investment in Computer Education since the first years of school. As robotics is one of the tools used to teach computational concepts, LEGO Robotics stands out in schools motivating research involving technology. Figure X shows the countries distribution where research has been done. Here, the number of studies is more than the total selected studies, because the same study could be carried out by authors from different countries. 
\item Publication vehicles: The studies were published in Conference (twenty-six studies), Symposium (five studies), Journal (four studies), and Workshop (one studies).
\item Types of studies:  The studies were also classified by the adopted method. Most of the selected studies presented the experiences with LEGO Robotics in teaching some subject. Few studies showed validation, but those who did, it was a comparative case study. The types of studies found in the selected papers were case study (twenty studies), experience report (fifteen studies), and experimental study (one study). We highlight the descriptive and comparative case studies were classified only as a case study. Figure X summarizes the types of studies. 
\end{enumerate}

\subsection{RQ1: What environments and programming languages have been used for teaching through LEGO Robotics?}

According to the selected studies, the programming environment most used for teaching though LEGO Robotics is LabVIEW (twenty-eight studies), considering the LEGO Mindstorm NXT, EV3 or both. The NXT version (NXT-G) was used by nineteen studies, whereas the EVX by seven studies, but two studies used both versions. LabVIEW is a development environment design for LEGO Mindstorm Education products that allows students and teachers programming robots through block-based language specific to LEGO robotics.

The Eclipse environment was cited by two studies, while Lejos, Enchanting, BricxCC, AIA, and AdaCore were cited only by one study each one. Besides, the study \cite{E23} did not mention which environment was used.  Table \ref{environments_used} summarizes the used environments in our SRL.

\begin{table}[h]
\renewcommand{\arraystretch}{1.3}
\caption{The environments used in the selected studies}
\label{environments_used}
\centering
\begin{tabular}{|c||p{3cm}||c|}
\hline \textbf{Environments} &\textbf{Study} & \textbf{Frequency}\\
\hline LabVIEW (NXT)	 & \cite{E03}, \cite{E04}, \cite{E09}, \cite{E11}, \cite{E15}, \cite{E20}, \cite{E21}, \cite{E24}, \cite{E25}, \cite{E26}, \cite{E27}, \cite{E28}, \cite{E30}, \cite{E31}, \cite{E32}, \cite{E33}, \cite{E34}, \cite{E35}, \cite{E36}    & 19\\
\hline LabVIEW (EV3) & \cite{E01}, \cite{E05}, \cite{E06}, \cite{E07}, \cite{E08}, \cite{E12}, \cite{E16} & 7\\
\hline LabVIEW (NXT / EV3) & \cite{E10}, \cite{E14} & 2\\
\hline Eclipse & \cite{E13}, \cite{E19} & 2\\
\hline Lejos & \cite{E22} & 1\\
\hline Enchanting & \cite{E18} & 1\\
\hline BricxCC & \cite{E02} & 1\\
\hline AIA & \cite{E29} & 1\\
\hline AdaCore & \cite{E17} & 1\\
\hline Not specified & \cite{E23} & 1\\
\hline
\end{tabular}
\end{table}

Figure X presents the frequency of the programming language that have been used for teaching through LEGO Robotic. The most programming language used was block-language (total of thirty studies), out of which twenty-eight studies were related to LabVIEW NXT and/or EV3 (\cite{E01}, \cite{E03}, \cite{E04}, \cite{E05}, \cite{E06}, \cite{E07}, \cite{E08}, \cite{E09}, \cite{E10}, \cite{E11}, \cite{E12}, \cite{E14}, \cite{E15}, \cite{E16}, \cite{E20}, \cite{E21}, \cite{E24}, \cite{E25}, \cite{E26}, \cite{E27}, \cite{E28}, \cite{E30}, \cite{E31}, \cite{E32}, \cite{E33}, \cite{E34}, \cite{E35}, \cite{E36} ), while only one study was related to App Inventor plus (\cite{E29}), finally, one study was related to Scratch (\cite{E18}). ScriptC language was used in two studies (\cite{E13}, \cite{E19}), whereas NXC (\cite{E02}), JAVA (\cite{E22}), C\# (\cite{E02}), ADA (\cite{E17}) language were used in one study each one. 

% An example of a floating figure using the graphicx package.
% Note that \label must occur AFTER (or within) \caption.
% For figures, \caption should occur after the \includegraphics.
% Note that IEEEtran v1.7 and later has special internal code that
% is designed to preserve the operation of \label within \caption
% even when the captionsoff option is in effect. However, because
% of issues like this, it may be the safest practice to put all your
% \label just after \caption rather than within \caption{}.
%
% Reminder: the "draftcls" or "draftclsnofoot", not "draft", class
% option should be used if it is desired that the figures are to be
% displayed while in draft mode.
%
%\begin{figure}[!t]
%\centering
%\includegraphics[width=2.5in]{myfigure}
% where an .eps filename suffix will be assumed under latex, 
% and a .pdf suffix will be assumed for pdflatex; or what has been declared
% via \DeclareGraphicsExtensions.
%\caption{Simulation results for the network.}
%\label{fig_sim}
%\end{figure}

% Note that the IEEE typically puts floats only at the top, even when this
% results in a large percentage of a column being occupied by floats.


% An example of a double column floating figure using two subfigures.
% (The subfig.sty package must be loaded for this to work.)
% The subfigure \label commands are set within each subfloat command,
% and the \label for the overall figure must come after \caption.
% \hfil is used as a separator to get equal spacing.
% Watch out that the combined width of all the subfigures on a 
% line do not exceed the text width or a line break will occur.
%
%\begin{figure*}[!t]
%\centering
%\subfloat[Case I]{\includegraphics[width=2.5in]{box}%
%\label{fig_first_case}}
%\hfil
%\subfloat[Case II]{\includegraphics[width=2.5in]{box}%
%\label{fig_second_case}}
%\caption{Simulation results for the network.}
%\label{fig_sim}
%\end{figure*}
%
% Note that often IEEE papers with subfigures do not employ subfigure
% captions (using the optional argument to \subfloat[]), but instead will
% reference/describe all of them (a), (b), etc., within the main caption.
% Be aware that for subfig.sty to generate the (a), (b), etc., subfigure
% labels, the optional argument to \subfloat must be present. If a
% subcaption is not desired, just leave its contents blank,
% e.g., \subfloat[].


% An example of a floating table. Note that, for IEEE style tables, the
% \caption command should come BEFORE the table and, given that table
% captions serve much like titles, are usually capitalized except for words
% such as a, an, and, as, at, but, by, for, in, nor, of, on, or, the, to
% and up, which are usually not capitalized unless they are the first or
% last word of the caption. Table text will default to \footnotesize as
% the IEEE normally uses this smaller font for tables.
% The \label must come after \caption as always.
%
%\begin{table}[!t]
%% increase table row spacing, adjust to taste
%\renewcommand{\arraystretch}{1.3}
% if using array.sty, it might be a good idea to tweak the value of
% \extrarowheight as needed to properly center the text within the cells
%\caption{An Example of a Table}
%\label{table_example}
%\centering
%% Some packages, such as MDW tools, offer better commands for making tables
%% than the plain LaTeX2e tabular which is used here.
%\begin{tabular}{|c||c|}
%\hline
%One & Two\\
%\hline
%Three & Four\\
%\hline
%\end{tabular}
%\end{table}


% Note that the IEEE does not put floats in the very first column
% - or typically anywhere on the first page for that matter. Also,
% in-text middle ("here") positioning is typically not used, but it
% is allowed and encouraged for Computer Society conferences (but
% not Computer Society journals). Most IEEE journals/conferences use
% top floats exclusively. 
% Note that, LaTeX2e, unlike IEEE journals/conferences, places
% footnotes above bottom floats. This can be corrected via the
% \fnbelowfloat command of the stfloats package.




\section{Conclusion}
The conclusion goes here.




% conference papers do not normally have an appendix


% use section* for acknowledgment
\section*{Acknowledgment}


The authors would like to thank...





% trigger a \newpage just before the given reference
% number - used to balance the columns on the last page
% adjust value as needed - may need to be readjusted if
% the document is modified later
%\IEEEtriggeratref{8}
% The "triggered" command can be changed if desired:
%\IEEEtriggercmd{\enlargethispage{-5in}}

% references section

% can use a bibliography generated by BibTeX as a .bbl file
% BibTeX documentation can be easily obtained at:
% http://mirror.ctan.org/biblio/bibtex/contrib/doc/
% The IEEEtran BibTeX style support page is at:
% http://www.michaelshell.org/tex/ieeetran/bibtex/
%\bibliographystyle{IEEEtran}
% argument is your BibTeX string definitions and bibliography database(s)
%\bibliography{IEEEabrv,../bib/paper}
%
% <OR> manually copy in the resultant .bbl file
% set second argument of \begin to the number of references
% (used to reserve space for the reference number labels box)

\bibliographystyle{abbrv}
\bibliography{referencias}

% that's all folks
\end{document}


